\documentclass[10pt,a4paper]{book}
\usepackage[margin=1in]{geometry} 
\usepackage[utf8]{inputenc}
\usepackage{amsmath}
\usepackage{amsthm}
\usepackage{amsfonts}
\usepackage{amssymb}
\usepackage{makeidx}
\usepackage{graphicx}
\usepackage[spanish]{babel}
\author{Ivan Leonardo Galeano S.}
\title{Estudio Estadística}

% ------------------------------------------------------------------------------
% Definitions (do not change this)
% ------------------------------------------------------------------------------
\newtheorem{defi}{\textbf{Definición}}
\newtheorem{teo}{\textbf{Teorema}}

\newcommand{\HRule}[1]{\rule{\linewidth}{#1}} 	% Horizontal rule

\makeatletter							% Title
\def\printtitle{%						
	{\centering \@title\par}}
\makeatother									

\makeatletter							% Author
\def\printauthor{%					
	{\centering \large \@author}}				
\makeatother							

% ------------------------------------------------------------------------------
% Metadata (Change this)
% ------------------------------------------------------------------------------
\title{	\normalsize \textsc{Notas de clase} 	% Subtitle of the document
	\\[2.0cm]													% 2cm spacing
	\HRule{0.5pt} \\										% Upper rule
	\LARGE \textbf{\uppercase{Estadística para ingenieros y científicos}}	% Title
	\HRule{2pt} \\ [0.5cm]								% Lower rule + 0.5cm spacing
	\normalsize \today									% Todays date
}

\author{
	Ivan Leonardo Galeano Saavedra\\	
	Estudio personal\\	
	Bogotá, Colombia\\
	\texttt{ilgaleanos@gmail.com} \\
}



\begin{document}
\thispagestyle{empty}
\printtitle
\vfill
\printauthor

\tableofcontents

\chapter{Muestreo y estadística descriptiva}

\section*{Introducción}

La recopilación y el análisis de datos son fundamentales en la ciencia y en la ingeniería; con esto los científicos descubre los principios que gobiernan el mundo físico y los ingenieros aprenden como diseñar nuevos productos y procesos.\\

La \textbf{estadística} se dedica a la recopilación, el análisis y la interpretación de datos con incertidumbre. Los \textbf{métodos} de la estadística permiten que los ingenieros y científicos diseñen experimentos válidos y obtengan conclusiones confiables a partir de los datos obtenidos.\\

La \textbf{idea básica} que yace en todos los métodos estadísticos de análisis de datos es inferir respecto de una población por medio del estudio de una muestra relativamente pequeña elegida de esta. Los métodos que se dedican a obtener conclusiones a partir de los datos constituyen el campo de la \textbf{estadística inferencial}. Los métodos que se dedican a recopilación de datos y a resumir claramente la información son del campo del \textbf{muestreo} y \textbf{estadística descriptiva}\\

\section{Muestreo}

Para entender la naturaleza de una muestra aleatoria simple, piense en una lotería. Imagine que se han vendido diez mil billetes y que se eligen cinco ganadores. ¿Cuál es la manera justa de elegir los ganadores? Es colocar los boletos en un recipiente, mezclarlos y extraer cinco de ellos uno tras otro.\\

\begin{defi}
	Una \textbf{población} representa la colección completa de elementos o resultados de la información buscada.
\end{defi}

\begin{defi}
	Una \textbf{muestra} constituye un subconjunto de una población, que contiene elementos o resultados que realmente se observan.
\end{defi}

\begin{defi}
	Una \textbf{muestra aleatoria simple} (\textit{MAS}) de tamaño $ n $ es una muestra elegida por un método en el que la selección de $ n $ elementos de la población tiene la misma probabilidad de formar la muestra, de la misma manera que en una lotería.
\end{defi}

En algunos casos, es difícil o imposible extraer una muestra de una manera realmente aleatoria. En esta situación, lo mejor que se puede hacer es seleccionar los elementos de la muestra por algún método conveniente. Por ejemplo, imagine que un ingeniero civil acaba de recibir una remesa de mil bloques de hormigón, que pesan aproximadamente 50 libras cada uno. Los bloques se han entregado en una gran pila. Para tomar una muestra aleatoria se requerirá sacar bloques del centro y de la parte inferior, lo que puede ser muy difícil. Por esta razón, el ingeniero puede tomar una muestra tomando diez bloques de la parte superior de la pila. Una muestra tomada así se llama muestra de conveniencia.\\

\begin{defi}
	Una \textbf{muestra de conveniencia} es aquella que no se extrae por un método aleatorio bien definido.
\end{defi}

El problema con las muestras de conveniencia es que podrían diferir sistemáticamente de la población en alguna forma. Por esta razón, tales muestras no se deben usar, excepto en situaciones donde \textbf{no es viable} tomar una muestra aleatoria.\\

Algunas personas piensan que una muestra aleatoria simple es garantía de que se refleja perfectamente a la población. Esto \textbf{no es cierto}. Las muestras aleatorias simples siempre son diferentes de sus poblaciones en algunos aspectos y en algunas ocasiones podrían ser considerablemente diferentes. Dos muestras diferentes de la misma población también diferirán entre si. Este fenómeno se conoce como \textbf{variación de muestreo}. Esta última constituye una de las razones por la cual los experimentos científicos tienen resultados diferentes cuando se repiten, a pesar de que las condiciones parezcan idénticas.\\

\begin{defi}
	Una \textbf{población tangible} es aquella que está conformada por elementos físicos reales. Estas poblaciones son siempre finitas.
\end{defi}

\begin{defi}
	Una muestra aleatoria simple puede consistir de valores obtenidos en un proceso en condiciones experimentalmente idénticas. En este caso, la muestra proviene de una población que consta de todos los valores posibles que se han observado. A este tipo de población se le denomina \textbf{población conceptual}.
\end{defi}

Los métodos estadísticos algunas veces se usan para mostrar que un conjunto de datos \textit{no} representa necesariamente una muestra aleatoria simple. Un método simple, pero efectivo para detectar esta condición, es realizar una gráfica con las observaciones en el orden en que se tomaron. Una muestra aleatoria simple no debe mostrar ningún patrón o tendencia obvia. Sin embargo antes de tomar esa decisión, es aún importante pensar acerca del proceso que produjo esos datos, ya que pueden haber cuestiones que no son evidentes en la gráfica.\\

\subsubsection{Independencia}

\begin{defi}
	Los elementos de una muestra son \textbf{independientes} si el conocimiento de algunos de los valores de los elementos no ayuda a predecir los valores de los otros.
\end{defi}

Los elementos en una muestra aleatoria simple se pueden tratar como independientes en muchos casos que se encuentran en la práctica. Ocurre una excepción cuando la población es finita y la muestra consiste de una parte importante (más de $ 5\% $) de la población.\\

\begin{defi}
	Se denomina \textbf{muestreo con remplazo}, si dada una población al tomar un elemento para la muestra, este es reemplazado dentro de la población antes de la toma del siguiente elemento.
\end{defi}

Con el muestreo con reemplazo la población siempre es exactamente la misma en cada extracción y los elementos muestreados son realmente independientes además que genera cualquier población se comporte como si fuera infinita.\\


\subsubsection{Otros métodos de muestreo}

\begin{defi}
	En el \textbf{muestreo ponderado} a algunos elementos se les da una mayor oportunidad que a los otros para ser seleccionados.
\end{defi}

\begin{defi}
	En el \textbf{muestreo aleatorio estratificado}, la población se divide en subpoblaciones, llamadas \textbf{estratos} y se extrae una muestra aleatoria simple de cada estrato.
\end{defi}

\begin{defi}
	En el \textbf{muestreo agrupado}, los elementos se extraen de la población en grupos o conglomerados.
\end{defi}
	
Ejemplo del muestreo agrupado. Puede dividir a toda la población (de un país) en diferentes conglomerados (ciudades). Luego, el investigador selecciona una serie de conglomerados en función de su investigación, a través de un muestreo aleatorio simple o sistemático. Luego, de los conglomerados seleccionados (ciudades seleccionadas al azar) el investigador puede incluir a todos los estudiantes secundarios como sujetos o seleccionar un número de sujetos de cada conglomerado a través de un muestreo aleatorio simple o sistemático.\\


\subsubsection{Tipos de experimentos}

\begin{defi}
	En un experimento de \textbf{una muestra} hay solamente una población de interés y se extrae una muestra de ésta.
\end{defi}

\begin{defi}
	En un experimento de \textbf{múltiples muestras} hay dos o más poblaciones de interés y se toma una muestra de cada una de ellas. 
\end{defi}

Generalmente se emplean múltiples muestras en por ejemplo: selección de dos o mas tipos de materiales\\

\begin{defi}
	En muchos experimentos de muestras múltiples las poblaciones se distinguen entre si al cambiar uno o más factores que pueden afectar el resultado. A estos se les llama \textbf{experimentos factoriales}.
\end{defi}

Por ejemplo medir la dureza ante el impacto de la muesca Charpy V para un importante número de soldaduras. Cada soldadura estaba hecha de uno de dos tipos de metales base y se había medido su dureza a diferentes temperaturas. Éste fue un experimento factorial con dos factores: el metal base y la temperatura.

El \textit{propósito} de un experimento factorial es determinar cómo afecta el resultado al cambiar los niveles de los factores.

\subsubsection{Tipos de datos}

\begin{defi}
	Cuando se asigna una cantidad numérica a cada elemento de una muestra, al conjunto de valores resultante se llama \textbf{numérico} o \textbf{cuantitativo}. En algunos casos, los elementos de la muestra son puestos en categorías. Entonces los datos son \textbf{categóricos} o \textbf{cualitativos}.
\end{defi}

\subsubsection{Ejercicios de la sección 1.1}

\begin{enumerate}
	\item Cada uno de los siguientes procesos implica el muestreo de una población. Defina la población y diga si es tangible o conceptual.
	\begin{enumerate}
		\item Se recibe una remesa de pernos de un distribuidor. Para verificar si es aceptable respecto de la fuerza de corte, un ingeniero selecciona diez pernos uno tras otro, del recipiente para probarlos.
		
		\textbf{R:} Población tangible: remesa de pernos.\\
		
		\item la resistencia de cierto resistor se mide cinco veces con el mismo óhmetro.
		
		\textbf{R:} Población conceptual: mediciones del resistor.\\
		
		\item Un estudiante de postgrado que se especializa en ciencia ambiental forma parte de un equipo de estudio que está evaluando el riesgo para la salud humana de cierto contaminante presente en el agua de la llave en su pueblo. Una parte del proceso de evaluación implica calcular la cantidad de tiempo que las personas que viven en este pueblo está en contacto con el agua de la llave. El estudiante convence a los residentes del pueblo para que lleven una agenda mensual, detallando la cantidad de tiempo que están en contacto con el agua de la llave día a día.
	
		\textbf{R:} Población tangible: personas en contacto con el agua.\\
		
		\item  Se hacen ocho soldaduras con el mismo proceso y se mide la fuerza en cada una de ellas.
		
		\textbf{R:} Población conceptual: soldaduras a medir\\
		
		\item  Un ingeniero responsable del control de calidad tiene que calcular el porcentaje de piezas fabricadas defectuosas en determinado día. A las 2:30 de la tarde muestrea las últimas 100 piezas fabricadas.
		
		\textbf{R:} Población tangible: piezas fabricadas.\\
	\end{enumerate}
	
	\item Si usted quisiera calcular la altura media de todos los estudiantes en una universidad, ¿cuál de las siguientes estrategias de muestreo sería la mejor? ¿Por qué? Observe que ninguno de los métodos son realmente muestras aleatorias simples.
	
	\begin{enumerate}
		\item Medir la estatura de 50 estudiantes que se encuentran en el gimnasio durante el juego de basquetbol en la escuela.
		\item Medir la estatura de todos los especialistas en ingeniería.
		\item Medir la estatura de de los estudiantes, eligiendo el primer nombre de cada página de la guía telefónica del campus universitario\\
		
		\textbf{R:} El que menos sesgo genera es el tercer método, esto debido a que es poco probable que el primer nombre de cada página de la guía telefónica del campus este relacionado directamente con un grupo poblacional específico. No es una \textit{MAS} puesto que posee un método sistemático al ser el primero de cada página y en un nuevo muestreo las población no tiene posibilidad de variar.
	\end{enumerate}
	
	\item Verdadero o falso:
	\begin{enumerate}
		\item Una \textit{MAS} garantiza que refleja exactamente a la población de la que se extrajo
		
		\textbf{R:} Falso.\\
		
		\item Una \textit{MAS} está libre de cualquier tendencia sistemática en diferir de la población de la que se extrajo
		
		\textbf{R:} Verdadero.
	\end{enumerate}
	
	\item Una ingeniera de control de calidad extrae una \textit{MAS} de $ 50 $ anillos de un lote de varios miles. Mide el espesor de cada uno y descuble que 45 de ellos, $ 90\% $, cumple con cierta especificación. ¿Cuál de los siguientes enunciados es correcto?
	
	\begin{enumerate}
		\item La proporción de anillos en el lote que cumple con la especificación probablemente es igual a $ 90\% $.
		
		\textbf{R:} Correcto.
		
		\item La proporción de anillos en el lote completo que cumple con la especificación probablemente está cerca de representar $ 90\% $, pero probablemente no es igual al total.
		
		\textbf{R:} Incorrecto.
	\end{enumerate}
	
	\item Se ha usado durante mucho tiempo un proceso para la fabricación de botellas de plástico y se sabe que $ 10\% $ de estas se encuentra defectuoso. Se está probando un nuevo proceso que, se supone, reduce la proporción de defectos. En una \textit{MAS} de $ 100 $ botellas producidas con el nuevo proceso $ 10 $ estaban defectuosas.
	
	\begin{enumerate}
		\item Uno de los ingenieros sugiere que la prueba demuestra que el nuevo proceso no es mejor que el anterior, ya que la proporción de defectos es la misma. ¿Es ésta una conclusión justificada? Explique.
		
		\textbf{R:} No, dado que lo importante es la proporción poblacional, adicionalmente la proporción de defecto es una aproximación que puede ser del $ 9\% $ o más.
		
		\item Suponga que hubieran sido solamente nueve las botellas defectuosas de la muestra de $ 100 $. ¿Esto habría probado que el nuevo proceso es mejor? Explique. 
		
		\textbf{R:} No, aunque la proporción muestral fue del $ 9\% $ la proporción para la población puede ser igual o superior de $ 10\% $.
		
		\item  ¿Qué resultado presenta pruebas más evidentes de que el nuevo proceso es mejor: encontrar nueve botellas defectuosas en la muestra o encontrar dos botellas defectuosas en la muestra?
		
		\textbf{R:} Encontrar dos botellas defectuosas.
	\end{enumerate}
	
	\item  Con referencia al ejercicio $ 5 $. Verdadero o falso:
	
	\begin{enumerate}
		\item Si la proporción de defectos en la muestra es menor a $ 10\% $, es confiable concluir que el nuevo proceso es mejor.
		
		\textbf{R:} Falso.\\
		
		\item Si la proporción de defectos en la muestra es sólo ligeramente menor a $ 10\% $, la diferencia bien podría ser completamente atribuible a la variación del muestreo y no es confiable concluir que el nuevo proceso es mejor.

		\textbf{R:} Verdadero.\\
		
		\item Si la proporción de defectos en la muestra en mucho menor a $ 10\% $, es muy poco probable que la diferencia sea atribuible a la variación del muestreo por lo que confiable llegar a la conclusión de que el nuevo proceso es mejor.

		\textbf{R:} Verdadero.\\
		
		\item No importa que tan pocos defectos aparezcan en la muestra, el resultado bien podría ser completamente atribuible a la variación del muestreo, por lo que no es confiable concluir que nuevo proceso es mejor.
		
		\textbf{R:} Falso.
	\end{enumerate}

	\item Para determinar si una muestra se debe tratar como \textit{MAS}, ¿qué es más importante un buen conocimiento de la estadística o un buen conocimiento del proceso que produce los datos?
	
	\textbf{R:} Un buen conocimiento del proceso que produce los datos.
	
\end{enumerate}

\section{Resumen estadístico}


Con frecuencia una muestra constituye una larga lista de números. Para ayudar a que las características de una muestra sean evidentes, se calcula el resumen estadístico. Las dos cantidades más usadas en el resumen son la \textbf{media de la muestra} y la \textbf{desviación estándar de la muestra}. La primera indica el centro de los datos y la segunda señala como están distribuidos los datos.

\begin{defi}
	Sea $ X_1, \ldots, X_n $ una muestra. La \textbf{media muestral} es $$ \overline{X} = \frac{1}{n} \sum_{i=1}^{n} X_i $$
\end{defi}

\begin{defi}
	Sea $ X_1, \ldots, X_n $ una muestra. la \textbf{varianza muestral} es la cantidad 
	$$ s^2 = \frac{1}{n-1} \sum_{i=1}^{n} (X_i - \overline{X})^2$$
	Una formula equivalente que puede ser mas fácil de calcular, es
	$$ s^2 = \frac{1}{n-1} \bigg{(} - n\overline{X} + \sum_{i=1}^{n} X^2_i \bigg{)}$$
\end{defi}

\begin{defi}
	Sea $ X_1, \ldots, X_n $ una muestra. La \textbf{desviación estándar muestral} es la cantidad
	$$ s = \sqrt{ \frac{1}{n-1} \sum_{i=1}^{n} (X_i - \overline{X})^2}$$
	Una formula equivalente que puede ser mas fácil de calcular, es
	$$ s = \sqrt{ \frac{1}{n-1} \bigg{(} - n\overline{X}^2 + \sum_{i=1}^{n} X^2_i \bigg{)} }$$
\end{defi}

\begin{defi}
	Sea $ X_1, \ldots, X_n $ una muestra y $ s $ la desviación estandar de la muestra. El \textbf{error estándar de la muestra} viene dado por $$ SE = \frac{s}{\sqrt{n}}$$
\end{defi}

Es natural preguntarse por qué la suma de las desviaciones al cuadrado se divide entre $ n-1 $ en lugar de $ n $. El propósito de calcular la desviación estándar muestral es calcular la cantidad de dispersión en la población de la cual se extrajo aquella. Por tanto, idealmente se calcularían las desviaciones de la media de todos los elementos de la población, en vez de las desviaciones de la media de la muestra. Sin embargo, la media de la población generalmente no se conoce por lo que en su lugar se usa la media de la muestra. Es un hecho matemático que las desviaciones alrededor de la media muestral tienden a ser un poco pequeñas que las desviaciones alrededor de la media poblacional y que al dividir entre $ n-1 $ en vez de $ n $ proporciona la rectificación correcta.

\begin{teo}\label{LinealidadDeLaMedia}
	Sea $ X_1, \ldots, X_n $ una muestra y $ Y_i = a + bX_i $, donde $ a $ y $ b $ son constantes, entonces $ \overline{Y} = a +b\overline{X} $.	
\end{teo}
\textbf{Dem:} Se sigue inmediatamente de la definición de media muestral.

\begin{teo}\label{LinealidadDeLaDesviacion}
	Sea $ X_1, \ldots, X_n $ una muestra y $ Y_i = a + bX_i $, donde $ a $ y $ b $ son constantes, entonces $s^2_Y= b^2 s^2_X$ y $s_Y = |b|s_X$
\end{teo}
\textbf{Dem:} Se sigue inmediatamente de la definición de varianza y desviación muestral y del teorema anterior.


\subsubsection{Datos atípicos}

A veces una muestra puede contener algunos puntos mucho más grandes o pequeños que el resto. Estos puntos se llaman datos \textbf{atípicos}. Estos datos representan un verdadero problema para los analistas de datos. como consecuencia de lo anterior, cuando las personas ven datos atípicos en sus datos tratan de encontrar una razón o un pretexto para eliminarlos. Sin embargo, un dato atípico, no se debe eliminar, a menos de que se tenga la certeza de que es resultado de un error. Cabe señalar que si una población realmente contiene datos atípicos y on eliminados de la muestra, esta ultima no caracterizará correctamente a la población.


\subsubsection{Mediana muestral}

La \textbf{mediana} al igual que la media representa una medida de tendencia central de los datos.

\begin{defi}
	Si $ n $ números están ordenados del más pequeño al mas grande, Si $ n $ es impar la mediana muestral es el numero en la posición $ \frac{n+1}{2}  $. Si $ n $ es par, la medina muestral representa el promedio de los números en las posiciones $ \frac{n}{2} $ y $\frac{n}{2} + 1$. 
\end{defi}

\subsubsection{La media recortada}

De la misma manera que la mediana, la \textbf{media recortada} es una medida de tendencia central que se diseño para que no esté afectada por datos atípicos. La media recortada se calcula al arreglar los valores de la muestra en orden, \textit{recortar} un número igual en cada extremo y calcular la media de los restantes. Si se \textit{recorta} el $ p\% $ de los datos de cada extremo, la media resultante se denomina \textit{media recortada un $ p\% $}; las medias más comunes de este tipo son donde $ p $ es $ 5 $, $ 10 $ y $ 20 $. Observe que la mediana se puede pensar como una forma extrema de la media recortada.\\

Debido a que el número de puntos de datos recortados debe ser un numero entero, e muchos casos es imposible recortar los porcentajes exactos que se piden de los datos. Si el tamaño muestral se denota por $ n $ y se desea recortar un $ p\% $, el número de datos a ser recortados es $ \frac{np}{100} $. Si este no es número entero redondeamos al entero más cercano.

\subsubsection{La moda y el rango}

La \textbf{moda} y el \textbf{rango} son resúmenes estadísticos de uso limitado, pero que en ocasiones se aprecian visualmente. La moda muestral es el valor que tiene más frecuencia en una muestra. Si algunos valores tienen una frecuencia igual, cada uno representa una moda. El rango es la diferencia entre los valores mas grandes y más pequeños en una muestra. Es una medida de la dispersión, pero rara vez se usa, porque depende solamente de los valores extremos y no proporciona ninguna información acerca del resto de la muestra.

\subsubsection{Cuartiles}

La mediana divide la muestra a la mitad. Los \textbf{cuartiles} la dividen tanto como sea posible en cuartos. Una muestra tiene tres de aquellos. Existen diferentes formas de calcular cuartiles, pero todas dan aproximadamente el mismo resultado. El método más simple cuando se calcula manualmente es el siguiente: Sea $ n $ el tamaño de la muestra. Ordene los valores de la muestra del más pequeño al más grande. Para encontrar el primer cuartil calcule $ 0.25(n+1) $, si éste es un número entero, entonces el valor de la muestra en esa posición es el primer cuartil de lo contrario sera el promedio de los valores de la muestra en cualquier lado de este valor. el segundo y tercer cuartil resultan de realizar el mismo procedimiento para los valores $ 0.5(n+1) $ y $ 0.75(n+1) $ respectivamente. Note que el segundo cuartil corresponde con el valor de la mediana.


\subsubsection{Percentiles}

el $p-$ésimo percentil de una muestra para un número p entre $ 0 $ y $ 100 $, divide a la muestra tanto como sea posible. Existen varios metodos para calcular percentiles pero mencionaremos uno similar al de calcular cuartiles, guiados por el valor de $ p(n+1)/100 $ si este es entero ese valor es el $p-$ésimo percentil de lo contrario es el promedio de los valores de la muestra en cualquier lado de este valor. Los percentiles se usan con frecuencia para interpretar puntajes de exámenes estandarizados. Por ejemplo si a una estudiante se le informa que su puntaje en un examen de ingreso a la universidad está en el $ 84 $avo percentil, esto significa que $ 84\% $ de los estudiantes que presentaron el examen obtuvo puntajes inferiores.


\subsubsection{Resumen estadístico para datos categóricos}

Con datos categóricos, a cada elemento de la muestra se le asigna una categoría en lugar de un valor numérico. Es necesario trabajar con datos categóricos y resúmenes numéricos. Los dos mas comunes son las \textbf{frecuencias} y las \textbf{proporciones muestrales} (algunas veces llamadas \textbf{frecuencias relativas}). La frecuencia para una categoría dada es sólo el número de elementos de la muestra que cae dentro de esa categoría. La proporción muestral es la frecuencia dividida entre el tamaño de la muestra.


\subsubsection{Ejercicios para la sección 1.2}

\begin{enumerate}
	\item Verdadero o falso: para cualquier lista de números, la mitad de ellos estará debajo de la media.\\
	\textbf{R:} Falso. Considere una muestra de valores iguales.\\
	
	\item ¿Es la media de la muestra siempre el valor que ocurre con mas frecuencia?. Si es así, explique por qué. Si no, dé un ejemplo.\\
	\textbf{R:} No, considere la muestra $ 1,1,3,3 $ la media es $2$ y no aparece en la muestra.\\
	
	\item ¿Es la media de la muestra siempre igual a uno de los valores que está en la muestra?. Si es así, explique por qué. Si no, dé un ejemplo.\\
	\textbf{R:} No, considere la muestra $ 1,1,3,3 $ la media es $2$ y no aparece en la muestra.\\
	
	\item ¿La mediana de la muestra siempre es igual a uno de los valores de la muestra? Si es así, explique por qué. Si no, dé un ejemplo.\\
	\textbf{R:} No, considere la muestra $ 1,1,3,3 $ la mediana es $2$ y no aparece en la muestra.\\
	
	\item Encuentre un tamaño de la muestra para el cual la mediana siempre sea igual a uno de los valores de la muestra.
	\textbf{R:} $ \{2 k - 1 \} $ donde $ k \in \mathbb{Z}^+ $.\\
	
	\item En cierta compañía, cada trabajador recibió un aumento de $ \$50 $ por semana. ¿Cómo afecta esto la media de los sueldos? ¿y la desviación estándar de los sueldos?\\
	\textbf{R:} Si $ \overline{x} $ es la media de los sueldos semanales entonces por el \textbf{teorema \ref{LinealidadDeLaMedia}} la nueva media esta dada por $\overline{x} + 50 $. De forma similar por el \textbf{teorema \ref{LinealidadDeLaDesviacion}} la desviación estándar no cambia.\\
	
	\item En otra compañía, cada trabajador recibió un aumento de $ 5\% $. ¿Cómo afecta esto la media de los sueldos? ¿y la desviación estándar de los sueldos?\\
	\textbf{R:} Si $ \overline{x} $ es la media de los sueldos semanales entonces por el \textbf{teorema \ref{LinealidadDeLaMedia}} la nueva media esta dada por $1.05\overline{x}$. De forma similar si $s$ es la desviación estándar por el \textbf{teorema \ref{LinealidadDeLaDesviacion}} la nueva está dada por es $ 1.05s $.\\
	
	\item El puntaje de Apgar se usa para evaluar reflejos y respuestas de recién nacidos. A cada bebé un profesional de la medicina le asigna un puntaje y los valores posibles son enteros entre cero y diez. Se toma una muestra de mil bebés nacidos en cierto condado y el numero con cada puntaje es el siguiente:\\
	
	\begin{tabular}{c|ccccccccccc}
	Puntaje	& 0 & 1 & 2 & 3 & 4 & 5 & 6 & 7 & 8 & 9 & 10 \\ 
		\hline 
	bebés & 1 & 3 & 2 & 4 & 25 & 35 & 198 & 367 & 216 & 131 & 18 \\ 
	\end{tabular} 
	
	\begin{enumerate}
		\item Encuentre la media de la muestra de los puntajes Apgar.\\
		\textbf{R:} $$ \overline{x} = 0,001\sum_{i=0}^{10} Puntaje_i \cdot Bebes_i = 7.138  $$

		\item  Encuentre la desviación estándar de la muestra de los puntajes Apgar.\\
		\textbf{R:} $$ s_x = \sqrt{\frac{1}{999}\bigg{(} - 1000\cdot7.138^2 + \sum_{i=0}^{10} Bebes_i \cdot Puntaje_i^2 \bigg{)} } = 1.311 $$
		
		\item Encuentre la mediana muestral de los puntajes de Apgar.\\
		\textbf{R:} Calculamos el promedio de las posiciones $500$ y $ 501 $. Como los puntajes están ordenados de $ 0 $ a $ 10 $, tenderemos que hacer sumas de la cantidad de bebes $ 1+3+2+4+25+35+198 = 268 + 367 = 635$ de donde la mediana es $ 7 $.
		
		\item  ¿Cuál es el primer cuartil de los puntajes?\\
		\textbf{R:} Esta ubicado en la posición $ 0.25(1000+1) = 250.25 $. Tendremos que es el promedio del valor en la posición $ 250 $ y $ 251 $, ademas recordemos que $ 1+3+2+4+25+35+198 = 268 $ de donde el primer cuartil es $ 6 $
		
		\item ¿Qué proporción de puntajes es más grande que la media?\\
		\textbf{R:} Sabemos que $ 216 +131 +18 = 365$ de donde $ 36.5\% $ son más grandes que la media.
		
		\item ¿Qué proporción de puntajes es mayor en una desviación estándar de la media?\\
		\textbf{R:} Sabemos que $ 7.138 + 1.311 = 8.449$ de donde son los bebes de $ 9 $ y $ 10 $ es decir $ 149 $ son mayores por lo tanto es de $ 14.9\% $
 
		\item ¿Qué proporción de puntajes está dentro de una desviación estándar de la media?\\
		\textbf{R:} Sabemos que $ 7 $ y $ 8 $ están incluidos, así que $ 7.138 - 1.311 = 5.827 $, estaría incluido el puntaje $ 6 $ es decir $ 781 $ bebes o lo que es el $ 78.1\% $
	\end{enumerate}

	\item Una clase de estadística con 40 estudiantes realizó una prueba. El puntaje posible más alto era de cuatro puntos. Diez estudiantes obtuvieron cuatro puntos, doce lograron tres puntos, ocho alcanzaron dos puntos, seis se beneficiaron con un  punto y cuatro obtuvieron cero puntos. Calcule la media, la mediana y la desviación estándar de los puntajes.
	
	\textbf{R:} $$ \overline{x} = 0.025\sum_{i=0}^{4} i \cdot NumEstudiantes_{puntaje_i} = 0.025 \cdot 98 = 2.45 $$
	
	$$ Mediana = 0.5 ( Posicion_{20} + Posicion_{21} ) = 0.5 ( 3 + 3 ) = 3 $$
	
	$$ s_x = \sqrt{\frac{1}{39} \bigg{(} - 40\cdot 2.45^2 + \sum_{0}^{4} NumEstudiantes_{puntaje_i} \cdot puntaje_i^2 \bigg{)}} =  1.299 $$
	
	\item Otra clase de estadística de 60 estudiantes realizó la misma prueba. En esta clase, 15 estudiantes obtuvieron cuatro puntos, 18 alcanzaron tres puntos, 12 lograron dos puntos, 9 obtuvieron un punto y 6 resultaron con cero puntos. Calcule la media, mediana y desviación estándar de los puntajes.
	
	\textbf{R:} $$ \overline{x} = \frac{1}{60} \sum_{i=0}^{4} i \cdot NumEstudiantes_{puntaje_i} = \frac{147}{60} = 2.45  $$
	
	$$ Mediana = 0.5 ( Posicion_{30} + Posicion_{31} ) = 0.5 ( 3 + 3 ) = 3 $$
	
	$$ s_x = \sqrt{\frac{1}{59} \bigg{(} - 60\cdot 2.45^2 + \sum_{0}^{4} NumEstudiantes_{puntaje_i} \cdot puntaje_i^2 \bigg{)}} =  1.294 $$
	
	
	\item En otra clase de estadística, el número total de estudiantes no se conoce. En esta clase, $ 25\% $ obtuvo cuatro puntos, $ 30\% $ alcanzó tres puntos, $ 20\% $ se benefició con dos puntos, $ 15\% $ logró un punto y $ 10\% $ resultó con dos cero puntos.
	
	\begin{enumerate}
		\item ¿Es posible calcular la media de los puntajes para esta clase? Si es así, calcúlela. Si no, explique por qué.\\
		\textbf{R:} $$ \frac{1}{k}\bigg{(}\frac{25}{100}k\cdot4 + \frac{30}{100}k\cdot3 + \frac{20}{100}k\cdot2 + \frac{15}{100}k\cdot1 + \frac{10}{100}k\cdot0 \bigg{)} = 2.45$$
		
		\item ¿Es posible calcular la mediana de los puntajes para esta clase? Si es así, calcúlela. Si no, explique por qué.\\
		\textbf{R:} Sin importar el tamaño de la muestra $ k $ la posición $ k/2 $ y $ 1+ k/n $ las cuales están contenidas en una vecindad del $ 50\% $ de la misma, contenida en el porcentaje que alcanzó $ 3 $ puntos.
		
		\item ¿Es posible calcular la desviación estándar de los puntajes para esta clase? Si es así, calcúlela. Si no, explique por qué.\\
		\textbf{R:} Asumiendo que calculamos, la desviación sesgada, dividimos por $k$. Si no, depende estrictamente del tamaño de la muestra. Luego de unos pasos se puede ver que
		$$s = \sqrt{-\overline{x}^2 + \frac{25}{100}16 +\frac{30}{100}9 +\frac{20}{100}4 + \frac{15}{100} } = 1.283 $$
	\end{enumerate}

	\item Cada uno de los $ 16 $ estudiantes mide la circunferencia una pelota de tenis por cuatro métodos diferentes, éstos fueron:
	
	Método A: Estimar la circunferencia a simple vista.
	
	Método B: Medir el diámetro con una regla y después calcular la circunferencia.
	
	Método C: Medir la circunferencia con una regla y cuerda.
	
	Método D: Medir la circunferencia haciendo rodar la pelota a lo largo de la regla.
	
	Los resultados (en cm) son los siguientes, en orden creciente para cada método:
	
	Método A: 18.0, 18.0, 18.0, 20.0, 22.0, 22.0, 22.5, 23.0, 24.0, 24.0, 25.0, 25.0, 25.0, 25.0, 26.0, 26.4.
	
	Método B: 18.8, 18.9, 18.9, 19.6, 20.1, 20.4, 20.4, 20.4, 20.4, 20.5, 21.2, 22.0, 22.0, 22.0, 22.0, 23.6.
	
	Método C: 20.2, 20.5, 20.5, 20.7, 20.8, 20.9, 21.0, 21.0, 21.0, 21.0, 21.0, 21.5, 21.5, 21.5, 21.5, 21.6.
	
	Método D: 20.0, 20.0, 20.0, 20.0, 20.2, 20.5, 20.5, 20.7, 20.7, 20.7, 21.0, 21.1, 21.5, 21.6, 22.1, 22.3.
	
	\begin{enumerate}
		\item Calcule la media de las mediciones para cada método.\\
		\textbf{R:} Método A = 22,74; Método B = 20.7, Método C = 21.01, Método D = 20.81
		
		\item Calcule la mediana de las mediciones para cada método.\\
		\textbf{R:} Método A = 23.5, Método B = 20.4, Método C = 21.0, Método D = 20.7
		
		\item Calcule la media recortada a $ 20\% $ para cada método.\\
		\textbf{R:} Recortaremos $ \lfloor 16\cdot20/100 \rfloor = 3$ datos de cada extremo. Método A = 23.25, Método B = 20.7, Método C = 21.04, Método D = 20.69
		
		\item Calcule el primer y tercer cuartil para cada método.\\
		\textbf{R:} Sabemos que $ .25(16+1) = 4.25 $ y $ .75(16+1) = 12.75 $.
		
		\textbf{Pimer Cuartil:} Método A = 21, Método B = 19.85, Método C = 20.75, Método D = 20.1
		
		\textbf{Tercer Cuartil:} Método A = 25, Método B = 22, Método C = 21.5, Método D = 21.3
		
		\item Calcule la desviación estándar de las mediciones para cada método.
		\textbf{R:}
		
		$$ s_{M_A} = \sqrt{\frac{1}{16+1}\bigg{(}-16\cdot22.74^2 + 8400.21\bigg{)}} = 2.727 $$
		
		$$ s_{M_B} = \sqrt{\frac{1}{16+1}\bigg{(}-16\cdot20.7^2 + 6883.32\bigg{)}} = 1.271 $$
		
		$$ s_{M_C} = \sqrt{\frac{1}{16+1}\bigg{(}-16\cdot21.01^2 + 7067.04\bigg{)}} = 0.504 $$
		
		$$ s_{M_D} = \sqrt{\frac{1}{16+1}\bigg{(}-16\cdot 20.81^2 + 6934.73\bigg{)}} = 0.586 $$
		
		\item ¿En qué método es la desviación estándar más grande? ¿Por qué se esperaría que este método tenga la desviación estándar más grande?.
		\textbf{R:} En el Método A, se esperaría ya que los datos no fueron capturados por un método fiable, se espera cambien abruptamente generando una mayor dispersión de los mismos.
		
		\item Sin que nada cambie ¿es preferible usar un método de medición que tenga una desviación estándar más pequeña o más grande? ¿O no importa? Explique\\
		\textbf{R:} Tratándose del estudio de métodos de medición la desviación estándar pequeña nos da un claro indicio de mayor precisión en las muestras indicando un mejor método. Aunque no es una regla infalible de elección  puesto que los de a simple vista pueden opinar todos el mismo valor haciendo que la desviación sea cero y no por eso es el mejor método.
		
		\end{enumerate}
	
	\item Con referencia al ejercicio anterior. 
	\begin{enumerate}
		\item Si las mediciones para cada uno de métodos se convirtieran a pulgadas (1 pulgada = 2.54cm), ¿cómo afectaría esto a la media? ¿a la mediana? ¿y los cuartiles? ¿Y la desviación estándar?\\
		\textbf{R:} Todo se escala con el valor de la conversión en este caso se multiplica por 2.54.
		
		\item Si los estudiantes midieran nuevamente la pelota, usando una regla marcada en pulgadas, ¿Los efectos sobre la media, la mediana, los cuartiles y la desviación estándar serían los mismos que los del inciso anterior? Explique.\\
		\textbf{R:} No serían exactamente los mismos dado que los datos recolectados serian diferentes al muestreo anterior.
	\end{enumerate}

	\item Una lista de diez número tiene una media de 20, una mediana de 18 y una desviación estandar de 5, el número mas grande de la lista es 39.27. Accidentalmente, este número se cambia a 392.7
	\begin{enumerate}
		\item ¿Cuál es el valor de la media después del cambio?\\
		\textbf{R:} la nueva media es de 55.343.
		\item ¿Cuál es el valor de la mediana despues del cambio?\\
		\textbf{R:} La mediana se mantiene en 18.
		\item ¿Cuál es el valor de la desviación estándar después del cambio?\\
		\textbf{R:} Supongamos $ x_i $ son las mediciones de la muestra y sabemos que  $$ s = \sqrt{\frac{1}{11}\bigg{(}-10\cdot20^2+ \sum x_i^2  \bigg{)}} = 5 $$ de donde obtenemos $$ \sum x_i^2 = 4275 $$ por lo tanto la nueva suma de cuadrados es $$ \sum y_i^2 = \sum x_i^2 - 39.27^2 + 392.7^2 = 4275 -1542.1329 + 154213.29 $$
		por lo tanto $$ \sum y_i^2 = 156790.6875 $$
		Así $$ s = \sqrt{\frac{1}{11}\bigg{(}-10\cdot55.343^2+ \sum y_i^2  \bigg{)}} = 107.09478 $$
	\end{enumerate}

	\item ¿Por qué nadie habla del cuarto cuartil? ¿O lo hacen?\\
	\textbf{R:} No existe un cuarto cuartil, existen tres cuartiles que dividen los datos en cuatro grupos iguales.
	
	\item En cada uno de los siguientes conjuntos de datos, diga si el dato atípico parece ser atribuible a un error, o si se podrá suponer correcto.
	\begin{enumerate}
		\item Una roca se pesa cinco veces. Las lecturas en gramos son 48.5, 47.2, 4.91, 49.5, 46.3.\\
		\textbf{R:} Es atribuido a un error es una variación muy amplia para el peso de una roca sin modificaciones de la misma.
		
		\item Un sociólogo muestrea cinco familias en cierto pueblo y registra sus ingresos anuales. Los ingresos son $ \$34000 $, $ \$ 57000 $, $ \$13000 $, $ \$1200000 $, $ \$62000 $\\
		\textbf{R:}  Los datos se puede suponer correctos.
	\end{enumerate}
	
	
\end{enumerate}

\end{document}           